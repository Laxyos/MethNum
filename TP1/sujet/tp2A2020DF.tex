%% LyX 2.3.4.2 created this file.  For more info, see http://www.lyx.org/.
%% Do not edit unless you really know what you are doing.
\documentclass[french]{article}
\usepackage[T1]{fontenc}
\usepackage[utf8]{inputenc}
\usepackage{babel}
\makeatletter
\addto\extrasfrench{%
   \providecommand{\og}{\leavevmode\flqq~}%
   \providecommand{\fg}{\ifdim\lastskip>\z@\unskip\fi~\frqq}%
}

\makeatother
\begin{document}
\title{TP Différences Finies 2020 MOCA 2A}
\maketitle

\section{Algorithme du Gradient conjugué }

\subsection{Cas-tests élémentaires}

Construire quelques exemples de systèmes linéaires $Ax=b$ de petite
taille où la matrice $A$ est symétrique définie positive. On pourra
par exemple écrire $A=BB^{t}$ avec $B$ inversible ou encore choisir
une matrice symétrique avec de grandes valeurs positives sur la diagonale
et des petites ailleurs.

\subsection{Programmation}

Programmer la méthode du gradient conjugué pour résoudre les systèmes
construits précédemment. On utilisera un test d'arrêt basé sur le
résidu.

\section{Différences finies pour le Laplacien 2D}

\subsection{Problème de la plaque}

Ecrire la méthode des différences finies pour résoudre l'équation
\[
-\triangle u(x,y)=0
\]
sur le domaine $D=[0,1]\times[0,1]$ avec comme conditions aux bords
de Dirichlet $u(x,1)=100$ et $u(x,y)=0$ sur les autres bords. Résoudre
le système obtenu à l'aide du gradient conjugué. Le produit $Ad_{k}$
sera calculé d'abord en ne tenant pas compte du caractère creux de
la matrice, puis en en tenant compte. On comparera les temps de calculs
en fonction du pas de discrétisation $h$ ainsi que l'erreur par rapport
à la solution exacte.

\subsection{Equation de Poisson}

Reprendre les questions précédentes pour l'équation
\[
-\triangle u(x,y)=-2\exp(x+y)
\]
avec conditions aux limites de Dirichlet $u(x,y)=\exp(x+y).$ On pourra
noter que la solution de cette équation est $u(x,y)=\exp(x+y).$

\section{Chaines de Markov pour le Laplacien 2D}

\subsection{Solution ponctuelle du problème de la plaque}

Construire un algorithme permettant de calculer la solution en un
point de la grille de pas $h$ à l'aide de $N$ simulations Monte-Carlo
de la chaine de Markov associé. Utiliser cet algorithme pour obtenir
la solution en tout point.

\subsection{Solution globale du problème de la plaque}

Construire un algorithme permettant un calcul de la solution globale
de l'équation en utilisant les états visités au cours des différentes
trajectoires. On choisira le point de départ au centre de la grille
ou uniformément au hasard dans la grille. Comparer avec la méthode
globale de la question 3.1.

\subsection{Sequential Monte Carlo pour la plaque}

Programmer $k$ étapes de la méthode Monte-Carlo séquentielle qui
consiste à calculer le résidu de l'équation par l'algorithme 3.2 et
à le rajouter à l'approximation à l'étape précédente. Comparer avec
l'algorithme 3.2 et au gradient conjugué du 2.1.

\subsection{Sequential Monte Carlo pour le problème de Poisson}

Programmer $k$ étapes de la méthode Monte-Carlo séquentielle et la
comparer avec le gradient conjugué.
\end{document}
